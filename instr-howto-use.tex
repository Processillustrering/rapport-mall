\section*{Hur ni använder detta malldokument}
Titta i källdokumentet för diverse inställningar för författare, titel, etc. Läs också käll\-doku\-men\-ten för instruktionerna i filerna \verb|instr-X.tex| för olika värden på \verb|X|.

\emph{OBSERVERA} att de ``fasta fält'' som blir på svenska (trots att ni ställt in engelska med \texttt{babel}), som Examinator, Handledare, datum på framsidan osv, \emph{ska} vara på svenska oavsett språk i rapporten. Abstract ska alltid vara på engelska, medan Sammanfattning alltid ska vara på svenska.

I flera appendix finns mer info som inte gäller rapportstrukturen.

För att slippa få med instruktionerna för rapportstrukturen i era inlämningar, ta bort \verb|\input{instr-X}| för alla värden av \verb|X|
i källdokumentet.

\textbf{Tips:} ni kan använda separata filer för de olika delarna i er rapport på motsvarande sätt, men använd inte samma filnamn!

\subsection*{Generellt}
Varje numrerat avsnitt ska finnas med i er slutrapport, om inget annat anges.  
Välj rubrik på svenska eller engelska beroende på ert valda rapportspråk.

Om ni skriver på engelska ska titeln skrivas med första bokstaven i varje ord versal, utom ``småord''. Exempel: \emph{A Really Interesting Project on the Fundamentals of Shoes}\footnote{Se t.ex.~\url{https://en.wikipedia.org/wiki/Capitalization\#Titles}}.  (Detta gäller även titlar i referenser på engelska.)
Rubrikerna i texten kan skrivas på detta sätt eller som på svenska (stor första bokstav i meningen), men \emph{var konsekventa}.

Glöm inte att läsa kurslitteraturen~\cite{dawson:projects-in-computing,dawson:projects-in-computing-old}.

% \subsection*{Uppdateringar av detta dokument}
% \begin{description}
% \item[2016-05-16]\mbox{}\\

% \end{description}


\section*{Att göra}
En sektion som beskriver läget för rapporten kan vara användbart i ``veckans inlämning'' för att underlätta feedbacken.

För att hantera ``att-göra-listor'' i rapporten kan La\TeX-paketet \verb|todonotes| kanske vara användbart. Se \url{http://ctan.org/pkg/todonotes} för mer info.


