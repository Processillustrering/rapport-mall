Här beskriver ni bakgrunden till ert projekt, d.v.s., det som leder fram till er problemformulering.  Vilket är området, omgivningen, kontextet, bakgrunden för projektet (mer detaljerat och djupare än i introduktionen)?  Beskriv området (t.ex. ljudbehandling, studieplaner, visualisering, autism...).  Beskriv uppdragsgivare, om ni har (men inte för detaljerat).  Tänk på att bakgrunden och problemet måste vara på en generell akademisk nivå och inte bara relaterat till en uppdragsgivare.

Tänk på att bakgrunden kan se längre tillbaka -- hur löste man problemet förr? Innan man började datorisera? Ibland är det både viktigt och intressant (men ibland inte).

Efter att ha läst bakgrunden ska det vara lätt att förstå syfte/mål och att de är viktiga. Ett typiskt bakgrundsavsnitt är 2-3 sidor.

%%% Local Variables:
%%% mode: latex
%%% TeX-master: "rapport-mall"
%%% End:
