Till\-känna\-gi\-van\-de eller Tack (Acknowledgement på engelska) behövs inte alltid, utan bara ibland: har man fått avsevärd hjälp/feedback/stöd från någon (utöver det som kan väntas t.ex. av en handledare), har man fått speciellt tillstånd att använda något material eller utrustning, har man fått ekonomisk sponsring (som t.ex. skulle kunna påverka den vetenskapliga opartiskheten)\ldots
Avsnittet ska vara kort och kärnfullt, typiskt ett stycke. Finns det behov att skriva något längre, diskutera med handledaren först.

